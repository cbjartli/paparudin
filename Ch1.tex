\subsection{Ex 1}
\textbf{Does there exist an infinite $\sigma$-algebra which has only countably
    many members?} 
\\

No. Assume $\Sigma$ is a countable $\sigma$-algebra on a space $X$, 
and for any $x \in X$ define 

\[ U_x = \bigcap_{\substack{U \in \Sigma \\ x \in U}} U \] 

Since $\sigma$-algebras are closed under countable intersection, $U_x \in \Sigma$.
Observe that given any $U \in \Sigma$, 

\[ U = \bigcup_{x \in U} U_x \]

so that $\Sigma$ is generated by $\{U_x\}$. 
\\

We further note that if $y \not \in U_x$, then $U_{y}^c$ is an element of $\Sigma$ that contains $x$, in which case $ U_x \subseteq U_y^c$, so that $U_x$ and $U_y$ are disjoint.
This implies that given two $x, y \in X$, then either $U_x$ and $U_y$ are equal, or they are disjoint.
\\

Since finite collections generate finite $\sigma$-algebras, $\{ U_x \}$ must be countably infinite. 
We can therefore assume that $ \{ U_x \} = \{ U_n \}_{n=1}^\infty$, and that $U_i$ and $U_j$ are disjoint whenever $i \neq j$.
\\

This implies that there exists an injection $2^\mathbb{N} \to \Sigma$, given by 
\[ (n_1, n_2, \cdots ) \mapsto \bigcap_{n_i = 1} U_{n_i} \in \Sigma \]
However, $|2^{\mathbb{N}}| > \aleph_0$, which contradicts the assumption that $\Sigma$ is countably infinite.


\subsection{Ex 2}
This proof is entirely analogous to the proof in the book. 
Let $f = (u_1(x), \cdots, u_n(x))$.
It suffices to show that $f$ is measurable.
Note that the product topology on $\mathbb R^n$ is generated by a countable union of products of the form $R = I_1 \times \cdots \times I_n$.  
Since $f^{-1}(R) = u_1^{-1}(R) \cap \cdots \cap u_n^{-1}(R)$ is measurable, we are done.
